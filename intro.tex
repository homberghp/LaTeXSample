\svnidlong{$HeadURL: https://www.fontysvenlo.org/svnp/879417/latexcolloquium/trunk/latexsample/intro.tex $}%
{$LastChangedDate: 2014-11-16 16:03:56 +0100 (zo, 16 nov 2014) $}%
{$LastChangedRevision: 62 $}%
{$LastChangedBy: 879417 $}
\svnid{$Id: intro.tex 62 2014-11-16 15:03:56Z 879417 $}

\renewcommand\TheFile{intro.tex}

% quote text
\begin{savequote}[8cm]
  \sffamily
  Introducing oneself properly is always hard.
  \qauthor{Anonymous}
\end{savequote}
\chapter{Introduction}
\pagenumbering{arabic}
\setcounter{page}{1}
One of the standards for documentation in open source and hence in
Linux land is \LaTeX, a text processing package. \LaTeX\ is available
for free and available with all Linux distributions and installed in
the Lab.

\TeX\ is the machinery of \LaTeX\ and was defined in the \TeX\ book
\citep{texbook} and implemented by
Prof. Donald Knuth. \LaTeX\ is a (nowadays HUGE) set of macros built
on top of that. \LaTeX\ in its initial form is described by Leslie
Lamport in \citep{latexbook}. If you like your book thick, try the
\LaTeX\ companion \citep{latexcompanion}.

The web is also a very good source of \LaTeX\ documentation. A good
starting point is \url{http://en.wikibooks.org/wiki/LaTeX}, useful for
beginners and pros alike.

This is a simple multi part document. It's purpose is to show how easy it is
to create a multi part document, one that, for instance, can be worked on 
simultaneously by several authors. Note that most of the settings for
this document are set in the file \texttt{mydefs.tex}. 
Look in that file too.

This document consists of the following files
\begin{enumerate}[noitemsep,topsep=0pt,parsep=0pt,partopsep=0pt]
\item \texttt{main.tex}
\item \texttt{motivation.tex}
\item \texttt{graphics.tex}
\item \texttt{codelisting.tex}
\item \texttt{poser.tex}
\item \texttt{hello.c}
\item \texttt{Makefile} and
\item \texttt{mydefs.tex}
\item \texttt{servlet3.png}, the pie chart
\item \texttt{Diagram1.pdf}, the UML diagram
\end{enumerate} 

You are kindly advised to keep your lab logs in simple text
files. These can be turned into latex files easily,
which can be used to produce a nice looking report.
\clearpage 
\section{Some hints to start with}
Sometimes things do not work out the way you think.
\LaTeX\ interpretes some character codes in it's own way.
Things like dollar signs or even underscore are special.
\LaTeX\ source are littered with accolades or \textit{curly braces} if that's
the way you call them. They are special too. So here is some advice: 

Do not use \define{funny file names}. That is: stick to ASCII filenames without spaces or even underscores. 
These will bring only you into trouble. If you want to keep things portable, 
don't use camel case (like in JavaClassNames) either, because
some OS-es do not distinguish between upper and lower case. You may of
course brake this rule if the files are program things like
Java source files.

\subsection{Hints for informatics (use version control)}
In software projects, versioning is important. \LaTeX\ and \textsc{SubVersion}
work nicely together here.
By using the \LaTeX\ package \texttt{svn-multi} you can have svn+
latex insert meta information about your files in the produced output,
for instance in the page footers, as in this document.

If you add the codes below at the top of each .tex file, these
codes will be expanded/updated by svn \textit{on checkin}. 
% do not expand keywords on  the example file
\lstinputlisting{svnkeywords.head}
You must also tell
subversion to expand these keywords for those files with for each file
the command:\\
{\lstset{language=sh,numbers=none,morekeywords={svn,propset,keywords,filename},keywordstyle={\color{blue}}}
\begin{lstlisting}[frame=single]
svn propset svn:keywords "Id Author File Date LastChangeDate
  Revision HeadURL Header"  filename
\end{lstlisting}
}
To keep these version codes up to date, first check if your \LaTeX\  files compile,
then check them in and do your final \LaTeX\  run. 

To make the version codes of the files appear on the bottom line, have
a look at the mydefs.tex file where the fancy headers and footers are defined.
In group work you may also find it comfortable to create a file named
\texttt{myauthors.tex} with contents like below, which is then
input into the main or mydefs at the proper place and can translate
student numbers from svn and you peerweb account into a humanly
readable authors name.

\lstinputlisting[frame=single]{myauthors.tex}

%%% Local Variables: 
%%% mode: latex
%%% TeX-master: "main"
%%% End: 
