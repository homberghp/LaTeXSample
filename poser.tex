\svnidlong{$HeadURL: https://www.fontysvenlo.org/svnp/879417/latexcolloquium/trunk/latexsample/poser.tex $}%
{$LastChangedDate: 2013-09-08 12:54:29 +0200 (Sun, 08 Sep 2013) $}%
{$LastChangedRevision: 31 $}%
{$LastChangedBy: 879417 $}
\svnid{$Id: poser.tex 31 2013-09-08 10:54:29Z 879417 $}
\renewcommand\TheFile{poser.tex}
\begin{savequote}[8cm]
  \sffamily
  Never start reading a difficult book at the wrong side.\\ 
  It makes you feel stupid.
  \qauthor{Private experience}
\end{savequote}
\chapter{So {\em you}\ know how to show off, but how do {\em I}\
  start?} 

\lstset{numberstyle={\tiny\color{red}},
  basicstyle={\small},numbers=right,
  rulesepcolor=\color{gray},
  showspaces=false,frame=shadowbox}

This document is indeed a bit of a showcase, but there is more to it.
In essence, most documents are mainly texts. 
And those plain texts take mainly typing and not much more.
The minimal, hello world style \LaTeX\ file is not much 
longer\footnote{Counting headers too!} then the C classic.
\lstinputlisting[firstline=5]{simple.tex}

And the pictures, well they are made with other packages, and as long
as those can produce a supported format, you can use them. \TeX\ and
\LaTeX\ have an own drawing language, but explaining that would blow
up this space.
There is a lot of documentation available on \TeX\ and \LaTeX\ and I
could recommend some good books on it.
But you need not run off to the nearest book shop. Lots of
documentation is on the Net, just as the \TeX\ program suite itself. 

Look for instance at \url{http://www.ntg.nl} for the Dutch \TeX\ user group
and \url{http://www.dante.de} for the German user group. They are both very 
alive and kicking.

A very good starting point nowadays is \url{http://en.wikibooks.org/wiki/LaTeX/}

\section{My own definitions}
Standard \LaTeX\ output looks a bit dull but are nicely formatted nonetheless.
If you want to make the most of your own style for your own or your
projects documentation, put your style definitions into a separate
file. That way you can keep all definitions of style and \define{macros} in one
spot. This works especially nicely in an multi part document, so
the other files (and authors) can concentrate on content.

You can \define{define} your own macros in \LaTeX. As an example a macro,
\verb#\define#, which I use
to let words stand out at the outer margin. I use it at the first use
of a word or concept in the text, to aid the reader in finding the
definition. Of course this macro could be extended to put the word
into an index. Which makes it a nice exercise.
The macro is defined as follows:\\
\lstinputlisting[frame=shadowbox,firstline=70,lastline=77]{mydefs.tex}

and is used as \verb#\define{new word}#.


%%% Local Variables: 
%%% mode: latex
%%% TeX-master: "main"
%%% End: 
