\svnidlong{$HeadURL: https://www.fontysvenlo.org/svnp/879417/latexcolloquium/trunk/latexsample/codelisting.tex $}%
{$LastChangedDate: 2014-11-16 16:03:56 +0100 (zo, 16 nov 2014) $}%
{$LastChangedRevision: 62 $}%
{$LastChangedBy: 879417 $}
\svnid{$Id: codelisting.tex 62 2014-11-16 15:03:56Z 879417 $}
\renewcommand\TheFile{codelistings.tex}

\begin{savequote}[8cm]
  \sffamily
  Use the source Luke.
  \qauthor{The most accurate documentation is in the source.}
  Addendum: Document your source well! And choose proper names.
  \qauthor{Pieter van den Hombergh}
\end{savequote}
\chapter{Listings and code documentation}
For these functions to work you need to use the \texttt{package} listings.
See mydefs.tex for the inclusion.

\lstset{%first some settings
  numbers=right, % number the lines
  numberstyle={\tiny\color{red}},frame=shadowbox,rulesepcolor=\color{gray},
  framexrightmargin=5mm
}

Note that the line numbers in the right hand border are the line
numbers in the included sources.

A good advice is to start using \define{doxygen} for your code documentation.
It can produce nicely formatted HTML and latex documents and can be
tuned in various ways with the  
help of doxywizard. The way of working is a lot like using javadoc,
but also works for C and C++ files. 
See e.g. the documentation on the zthreads package at
\url{http://zthread.sourceforge.net} or Qt at 
\url{http://doc.trolltech.com/3.3/index.html}

\section{Source code}
The most simple case: include the whole thing with a command like \\
\verb#\lstinputlisting[language=java]{code/Hi.java}#
\lstinputlisting[language=java]{code/Hi.java}

Sometimes it is useful to include just a part of a file, for instance
when you want to explain things. Like what line 11 is all about.\\
\verb#\lstinputlisting[language=java, firstline=11,lastline=11]{Hi.java}#
\lstinputlisting[language=java, firstline=11,firstnumber=11,
        lastline=11,numbers=right,basicstyle={\small\ttfamily},
        caption={use toString to let an object represent itself.}
        ]{code/Hi.java}

\section{Makefiles}
You can also include make files.
Note that makefiles have a peculiar syntax. \define{Spaces and tabs in
  Makefiles}\ are meaningful. 
That's why I made them show up in the next listing with the command\\
\lstinputlisting[firstline=65,lastline=67]{codelisting.tex}
\lstset{showspaces=true,showtabs=true}
Spaces show up as \lstinline| |, tab characters as an extended version
of the same thing (\lstinline|	|). 
As can be expected, spaces and tabs have no special meaning in
makefile comments, the lines starting with a hash (\#) sign. 
If you want to know more on makefiles try google or info:make in
konqueror on a decently installed Linux box.  

The make file for this entire document looks like this:
\lstinputlisting[language=make,showtabs=true,
     showspaces=true,basicstyle={\ttfamily\scriptsize},
     numbers=right,language=make]{Makefile}


%%% Local Variables: 
%%% mode: latex
%%% TeX-master: "main"
%%% End: 
